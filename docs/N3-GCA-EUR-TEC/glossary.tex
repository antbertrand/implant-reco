% Glossary and acronyms management
% See https://en.wikibooks.org/wiki/LaTeX/Glossary#Defining_symbols
% for more information.
%
% La forma de definir un acrónimo es la siguiente:
% \newacronyn{id}{siglas}{descripción}
% Donde:
% 	'id' es como vas a llamarlo desde el documento.
%	'siglas' son las siglas del acrónimo.
%	'descripción' es el texto que representan las siglas.
%
% Para usarlo en el documento tienes 4 formas:
% pepper your writing with \gls{mylabel} macros (and similar) to simultaneously insert your predefined text and build the associated glossary.
% \glsentryshort{id} - Añade solo las siglas de la id
% \glsentrylong{id} - Añade solo la descripción de la id
% \glsentryfull{id} - Añade tanto  la descripción como las siglas

\newacronym{n3}{N3}{NumeriCube}
\newacronym{cli}{CLI}{Command Line Interface}

\newglossaryentry{github}
{
    name={GitHub},
    description={
        The source management service (owned by Microsoft).
        See \url{https://www.github.com}
    }
}

\newglossaryentry{draft}
{
    name={Draft},
    description={
        A draft document is a document that is neither production-grade nor
        delivery-grade. A document whose reference is postfixed by the "\-DRAFT"
        mention is a document that has been generated during development process
        with misalignment between the repository revision and the moment it's been
        generated.
    }
}

% \newglossaryentry{real number}
% {
%   name={real number},
%   description={GROUMPF. include both rational numbers, such as $42$ and
%                $\frac{-23}{129}$, and irrational numbers,
%                such as $\pi$ and the square root of two; or,
%                a real number can be given by an infinite decimal
%                representation, such as $2.4871773339\ldots$ where
%                the digits continue in some way; or, the real
%                numbers may be thought of as points on an infinitely
%                long number line},
%   symbol={\ensuremath{\mathbb{R}}}
% }

\newglossaryentry{docker}
{
  name={Docker},
  description={include both rational numbers, such as $42$ and
               $\frac{-23}{129}$, and irrational numbers,
               such as $\pi$ and the square root of two; or,
               a real number can be given by an infinite decimal
               representation, such as $2.4871773339\ldots$ where
               the digits continue in some way; or, the real
               numbers may be thought of as points on an infinitely
               long number line},
  symbol={\ensuremath{\mathbb{R}}}
}
