%!TeX root = ../main.tex
\chapter{Entraînement}

Dans ce chapitre, vous retrouverez en détails les procédures pour l'entrainements des fichiers de poids Tiny Yolo.

\section{Yolo et Tiny Yolo sous Keras}

Le code des entraînements se trouve dans le dossier gcaesthetics-implantbox/train/keras-yolo3-master.
Les entrainements se font classiquement avec l'envoi et l'execution d'un container Docker sur Paperspace.
Les fichiers importants a modifier pour lancer/customiser les entraînements sont :

\subsection{run.sh}
Comme dit précement, le lancement de l'entraînement se fait sur container Docker, envoyé sur des machines Paperspace.
Le fichier run.sh va executer les commandes ligne par ligne.

\subsection{train.py}
C'est le script de lancement de l'entraînement.
Concrètement, les seuls paramètres a modifier dans ce fichiers sont :
\begin{itemize}
    \item anchors_path : Chemin d'accés du fichiers d'anchors.
    \item classes_path : Chemin d'accés du fichier de classes.
    \item input_shape : Taille des "input", comprendre la taille des photos en entrée. Cette valeurs doit forcément être un multiple de 32.
    \item Les paramètres d'entrainements plus bas dans le fichier au besoin (comme le batch_size, l'archi, etc...)
\end{itemize}

\subsection{convert_to_voc.py}
Permet de convertir les annotations from supervisely to keras yolo dans un fichier de sortie /artifact/train.txt, qui sera utilisé dans train.py.
Il est généré avant chaque lancement d'entraînement.
Il prends en entrée :
\begin{itemize}
    \item image_path : Chemin d'accés du dossier d'image d'un dataset donné.
    \item annot_path : Chemin d'accés du dossier d'annotations d'un dataset donné.
\end{itemize}

\subsection{Datasets}
Tous les datasets sont disponibles dans le bucket S3.
Ils sont synchronisés dans la partition /storage/.
Evidemment, tous ces datasets sur le projet Eurosilicone sont provisoires, et devront être complétés.

Pour des raisons évidentes de "rapiditée", les datasets ont été constitués des photos redimensionnées, en 10920*1080. (Format image provenant de la caméra : 5472*3648).
Par la suite, il serait donc opportun de constituer le dataset avec des photos "plein format" (5472*3648), au maximum de la résolution de la caméra, pour obtenir les images de pastille les plus "propres" possible.

\subsubsection{Détection des pastilles}
\begin{itemize}
    \item pastilles : dataset constitué avec la 1ère caméra couleur, pour la détection de pastilles, sans égalisation ni cropping/redressement.
\end{itemize}

\subsubsection{Détection de textes}
\begin{itemize}
    \item textes : dataset constitué avec la 1ère caméra couleur/focale, redressées.
    \item textes_equal : dataset « textes », avec les images égualisées.
    \item textes_20mpx : dataset constitué avec la caméra monochrome 20mpx, images redressés.
    \item textes_20mpx_equal : dataset « textes_20mpx », contenant les même images égualisées.
\end{itemize}

\subsubsection{Détection de caractères}
\begin{itemize}
    \item lettres_equal : dataset constitué avec la caméra monochrome 20mpx. Comprend les images croppées et redréssées autour des pastilles.
    \item lettres_equal_crop : dataset « lettres_equal », avec cropping autour des textes.
\end{itemize}

\subsection{Resultats de l'entraînement}

En sortie d'entraînement, nous avons un fichier de poids .h5
Ce fichier est généré dans le dossier /artifacts sur Paperspace.

\subsection{Evaluation de l'entraînement}

Pour la phase d'evaluation, nous allons executer l'inférence sur un dossier de photos jamais vues par l'algorithme.
Pour des questions pratiques de developpement, j'execute les inferences en local (commande : make debug), mais il est possible de l'executer sur Paperspace a la suite d'un entrainement...

\subsection{yolo_video.py}

Fichier permettant l'execution des inference. Les commandes a lancer sont :
\begin{lstlisting}[style=console]
    Air-de-Corentin:keras-yolo3-master cdidriche$ : make debug
    root@75120bff2ed0:/paperspace# : python3 yolo_video.py --image
    root@75120bff2ed0:/paperspace# : ./images_to_infer
    [.......]
\end{lstlisting}

\begin{itemize}
    \item make debug : Execution du container en local
    \item python3 yolo_video.py --image : Permet de lancer le script d'inférence
    \item ./images_to_infer : est le path contenant les images a inférer. Le script va boucler sur toutes les images du dossier en question. Un fichier inference_result.txt est généré dans ./model_data. Il va nous servir pour le calcul du mAP derrière, et donc, d'avoir une métrique d'évaluation significative
\end{itemize}

Par défault, les chemins sont en "dur", un dossier model_data qui contient les fichiers .h5, classes.txt, anchors.txt dans le fichier yolo.py.
C'est pas très smart ni user-friendly, mais au moins ca permet de ne pas faire d'erreurs de dossier/fichiers quand on utilise un même detecteur pour 3 cas différents...

\subsection{Calcul du mAP}

Pour calculer le mAP, on utilise une série de script qui permet d'avoir une evaluation "générique".
Le code du calcul de mAP se situe dans ./train/mAP

Avant de lancer le calcul du mAP, il faut génerer pour chaque image, un fichier "ground-truth" et un fichier "predicted objects".
Pour les générer simplement, un scripts sont a disposition pour convertir les annotations : convert_keras-yolo3.py
\begin{itemize}
    \item Depuis le fichier "train.txt" (sortie du script convert_to_voc.py), vers le dossier "ground-truth".
    \item Depuis le fichier "inference_result.txt" (sortie du script d'inference Yolo), vers le dossier "predicted".
\end{itemize}

Le script pour fonctionner a besoin des fichiers :
\begin{itemize}
    \item class_list.txt : Fichier contenant la liste des classes.
    \item train.txt OU inference_result.txt : Fichier contenant les annotations.
\end{itemize}

\begin{lstlisting}[style=console]
    Air-de-Corentin:mAP-master cdidriche$ cd extra/
    Air-de-Corentin:extra cdidriche$ python3 convert_keras-yolo3.py -o ../ground-truth/ -r --gt ../ground-truth/train.txt
    [.......]
    Air-de-Corentin:extra cdidriche$ python3 convert_keras-yolo3.py -o ../predicted/ -r --pred ../predicted/inference_result.txt
    [.......]
    Air-de-Corentin:mAP-master cdidriche$ python3 main.py
    100.00% = m AP
    100.00% = text AP
    mAP = 100.00%
\end{lstlisting}

En sortie, on dispose de tout un ensemble de graphique et meusure representant le mAP globale, par classe, etc...
Ces résultats sont diponibles et consultables dans le dossier /mAP-master/results
