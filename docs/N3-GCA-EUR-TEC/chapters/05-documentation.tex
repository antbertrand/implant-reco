%!TeX root = ../main.tex
\chapter{Documentation}

Documentation should be a whole part of your project.

Here are procedures regarding documentation authoring, expanding and generation.

\section{General considerations}

\subsection{Overall documentation workflow}

% Workflow for development activites
\begin{tikzpicture}[node distance=1.5cm,
    every node/.style={fill=white, font=\sffamily}, align=center]

    \node (init)                [base, green]
        {Prepare your repository for doc activity};
    \node (write)               [base, below of=init]
        {Write documentation (tex files)};
    \node (testCode)            [base, below of=write]
        {Test your documentation locally};
    \node (commitCode)          [base, below of=testCode]
        {Commit your code and push it to \gls{github}};
    \node (release)             [base, below of=commitCode]
        {Release your project};
    \node (generate)            [base, below of=release]
        {Your documentation is generated by your CI environment};

    % % Specification of lines between nodes specified above
    % % with aditional nodes for description
    \draw[->]         (init) -- (write);
    \draw[->]         (write) -- (testCode);
    \draw[->]         (testCode) -- (commitCode);
    \draw[->]         (commitCode) -- (release);
    \draw[->]         (release) -- (generate);
    \draw[->] (generate.east) -- ++(2.6,0) -- ++(0,2) -- ++(0,2) --
       node[xshift=1.2cm,yshift=-1.5cm, text width=2.5cm]
       {Rinse, repeat}(init.east);
\end{tikzpicture}

% \begin{tikzpicture}[node distance=1.5cm,
%     every node/.style={fill=white, font=\sffamily}, align=center]
%   % Specification of nodes (position, etc.)
%   \node (start)             [activityStarts]              {Activity starts};
%   \node (onCreateBlock)     [process, below of=start]          {onCreate()};
%   \node (onStartBlock)      [process, below of=onCreateBlock]   {onStart()};
%   \node (onResumeBlock)     [process, below of=onStartBlock]   {onResume()};
%   \node (activityRuns)      [activityRuns, below of=onResumeBlock]
%                                                       {Activity is running};
%   \node (onPauseBlock)      [process, below of=activityRuns, yshift=-1cm]
%                                                                 {onPause()};
%   \node (onStopBlock)       [process, below of=onPauseBlock, yshift=-1cm]
%                                                                  {onStop()};
%   \node (onDestroyBlock)    [process, below of=onStopBlock, yshift=-1cm]
%                                                               {onDestroy()};
%   \node (onRestartBlock)    [process, right of=onStartBlock, xshift=4cm]
%                                                               {onRestart()};
%   \node (ActivityEnds)      [startstop, left of=activityRuns, xshift=-4cm]
%                                                         {Process is killed};
%   \node (ActivityDestroyed) [startstop, below of=onDestroyBlock]
%                                                     {Activity is shut down};
%   % Specification of lines between nodes specified above
%   % with aditional nodes for description
%   \draw[->]             (start) -- (onCreateBlock);
%   \draw[->]     (onCreateBlock) -- (onStartBlock);
%   \draw[->]      (onStartBlock) -- (onResumeBlock);
%   \draw[->]     (onResumeBlock) -- (activityRuns);
%   \draw[->]      (activityRuns) -- node[text width=4cm]
%                                    {Another activity comes in
%                                     front of the activity} (onPauseBlock);
%   \draw[->]      (onPauseBlock) -- node {The activity is no longer visible}
%                                    (onStopBlock);
%   \draw[->]       (onStopBlock) -- node {The activity is shut down by
%                                    user or system} (onDestroyBlock);
%   \draw[->]    (onRestartBlock) -- (onStartBlock);
%   \draw[->]       (onStopBlock) -| node[yshift=1.25cm, text width=3cm]
%                                    {The activity comes to the foreground}
%                                    (onRestartBlock);
%   \draw[->]    (onDestroyBlock) -- (ActivityDestroyed);
%   \draw[->]      (onPauseBlock) -| node(priorityXMemory)
%                                    {higher priority $\rightarrow$ more memory}
%                                    (ActivityEnds);
%   \draw           (onStopBlock) -| (priorityXMemory);
%   \draw[->]     (ActivityEnds)  |- node [yshift=-2cm, text width=3.1cm]
%                                     {User navigates back to the activity}
%                                     (onCreateBlock);
%   \draw[->] (onPauseBlock.east) -- ++(2.6,0) -- ++(0,2) -- ++(0,2) --
%      node[xshift=1.2cm,yshift=-1.5cm, text width=2.5cm]
%      {The activity comes to the foreground}(onResumeBlock.east);
%   \end{tikzpicture}


\subsection{Definitions}

In the context of this document, 'Documentation' means written and printable documentation,
outside code, that's meant to be used as a reference for the project(s).

\begin{description}
    \item[SRS]: Software Requirement Specifications
    \item[DST]: Dataset description and management
    \item[TEC]: Technical documentation (specifications, ...)
    \item[TST]: Test procedure (test plans)
\end{description}

\subsection{Documentation reference (\texttt{docref})}

Each of your documentation name should be of the following format: \textbf{N3-CUS-PRJ-TYP} where:

\begin{description}
    \item[N3] are the letters "N3" \gls{n3} standing for NumeriCube.
    \item[CUS] is a trigram representing customer's name.
    \item[PRJ] is a trigram representing the project name.
    \item[TYP] is a trigram representing the documentation type (see above for examples).
\end{description}

\subsection{Documentation format}

Documentation should be written exclusively in \LaTeX with the style and format
of this present document. You shall find tutorials and other resources
in other parts of this document.

You can also grab and read the \LaTeX source code in order to understand how it works.


\section{Tooling}

The development tools that are validated with the current document are:

\begin{itemize}
    \item Atom (text editor, version >= 1.34.0)
    \item \texttt{latex} plugin, version >= 0.50.2
    \item \texttt{latex-syntax-highlighting}
\end{itemize}

Install \LaTeX locally on your Mac (see the procedure in the official documentation)

Configure these specifics for the plugin:

\begin{description}
    \item[TeX Path]: \texttt{/Library/TeX/texbin}
    \item[Engine]: lualatex
    \item[Enable Extended Build Mode]: yes
    \item[Output directory]: \texttt{build}
\end{description}

Configure the rest to your tastes. Use the shortcuts (\texttt{control} + \texttt{option} + \texttt{B} to build, for example).

Depending on your options, the PDF you're working on will open automatically.

\section{Writing documentation}

\subsection{Where to put your documentation}

In your \texttt{github}\gls{github} project,
put your documentation at the root of your project in a \texttt{docs} folder.

Each of your documentation item should have a \textbf{reference} (see above)
and be put in the \texttt{docs/}\textit{docref} directory of your project.

Set your \texttt{.gitignore} file so that generated files are never
collected into your code repostiory.

\subsection{Build the documentation structure}

Here's the procedure.

\begin{itemize}
    \item Copy the TEC documentation provided with \texttt{dmake} and put it in your folder.
    \item Edit the \texttt{main.tex} file and adjust to your needs.
    \item Edit the \texttt{chapters/*} files as you wish (same for other sections).
    \item Make sure you have a symlink between your doc folder and the \texttt{dmake/docs/include} folder (the symlink should not point elsewhere).
\end{itemize}

\subsubsection{Folders organization}

Your document directory should have the following layout:

\begin{description}
\item[main.tex]: At the root of your doc folder. This is what describes your document.
\item[annexes]: Tex content of the annexes of your doc. Some will be auto-generated.
\item[chapters]: This is where the real works happens :)
You can organize your chapters the way you want to but we recommend a directory per chapter.
\item[forewords]: Same as annexes but for introductory words.
\item[include]: Now this is where we define the document layout and appearance.
It should either be copied/pasted from a central repository ot whatever.
\end{description}

\begin{forest}
  for tree={font=\sffamily, grow'=0,
  folder indent=0.2em, folder icons,
  edge=densely dotted}
  [Project folder
    [...]
    [docs
        [N3-CUS-PRJ-TYP
            [main.tex, is file]
            [acronyms.tex, is file]
            [include]
            [forewords
                [01-intro.tex, is file]
            ]
            [chapters
                [01-chapter1.tex, is file]
                [02-chapter2.tex, is file]
            ]
            [annexes
                [01-annex1.tex, is file]
            ]
            [images
                [...]
            ]
        ]
    ]
  ]
\end{forest}

Some noteworthy considerations here:

\begin{itemize}
    \item \texttt{main.tex} is your main source file. Use from the examples given in Dmake source code,
        adapt it, make it shine. You'll have to set a few variables inside to make it work.
    \item \texttt{acronyms.tex} is where you store acronyms for your project
        (a few examples are included). The glossary will be automagically generated.
    \item \texttt{forewords}, \texttt{chapters} and \texttt{annexes} are the places where
        your content will be located. Try to make 1~file per chapter to keep stuff clear.
    \item \texttt{images} is your attachments folder.
    \item \texttt{include} is a symlink to \texttt{dmake/docs/include}
\end{itemize}

\section{Generating documentation}

\subsection{Generate all doc in your project}

From your project home, simply run:

\begin{lstlisting}[style=bash]
./dmake doc
\end{lstlisting}

The place your documentation is generated in is indicated in the returned information.

\begin{lstlisting}[style=console]
    [635][pjgrizel.MacBook-Pro-2017-PJ: dmake]$ ./dmake.py doc
    Preprocessing N3-INT-DMK-TEC
    Running LaTeX in Docker. This may take time (large container)
    [.......]
    If compiled successfully, your document is ready at: /Users/pjgrizel/Projects/n3-demos/provision/dmake/docs/build/N3-INT-DMK-TEC-6ad64ab-DRAFT.pdf
\end{lstlisting}


\section{\LaTeX conventions and usages}

This section if for die-hard \LaTeX fans who will know which modules to use (or not).

Documentation is built using a \texttt{latexmk} and \texttt{lualatex} pipeline.

Schemata are built with \texttt{TikZ}. You can find example schemas here: \url{http://www.texample.net/tikz/examples/tag/diagrams/}

The overall structure, macros and front cover are inspired from Universidad of Alicante design open-sourced here: \url{https://github.com/jmrplens/TFG-TFM_EPS}.
