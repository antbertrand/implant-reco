% !TeX encoding = UTF-8
% !TeX spellcheck = en_EN
% !TeX program = xelatex
% !TeX TXS-program:compile = txs:///xelatex/[--shell-escape]

%%%%%%%%%%%%%%%%%%%%%%%%%%%%%%%%%%%%%%%%%%%%%%%%%%%%%%%%%%%%%%%%%%%
%%%             GLOBAL PARAMETERS AND INCLUDES                  %%%
%%%%%%%%%%%%%%%%%%%%%%%%%%%%%%%%%%%%%%%%%%%%%%%%%%%%%%%%%%%%%%%%%%%

% Language (french, english, spanish)
\def\defaultlang{french}

% The magic config stuff
\input{include/config}

%%%%%%%%%%%%%%%%%%%%%%%%%%%%%%%%%%%%%%%%%%%%%%%%%%%%%%%%%%%%%%%%%%%
%%%          ADDITIONAL / CUSTOM CONFIGURATION                  %%%
%%%%%%%%%%%%%%%%%%%%%%%%%%%%%%%%%%%%%%%%%%%%%%%%%%%%%%%%%%%%%%%%%%%

\usepackage{csvsimple}

% http://www.texample.net/tikz/examples/android/
\usetikzlibrary{arrows.meta}
\tikzset{%
  >={Latex[width=2mm,length=2mm]},
  % Specifications for style of nodes:
            base/.style = {rectangle, rounded corners, draw=black,
                           minimum width=4cm, minimum height=1cm,
                           text centered, font=\sffamily},
            blue/.style = {base, fill=blue!30},
             red/.style = {base, fill=red!30},
           green/.style = {base, fill=green!30},
        ttfamily/.style = {base, minimum width=2.5cm, fill=orange!15,
                           font=\ttfamily},
}


%%%%%%%%%%%%%%%%%%%%%%%%%%%%%%%%%%%%%%%%%%%%%%%%%%%%%%%%%%%%%%%%%%%
%%%                  DOCUMENT INFORMATION                       %%%
%%%%%%%%%%%%%%%%%%%%%%%%%%%%%%%%%%%%%%%%%%%%%%%%%%%%%%%%%%%%%%%%%%%
% Except where indicated, all variables are mandatory.
% doctitle = title (document title / document type)
% project = PROJECT
\newcommand{\doctitle}{Documentation technique}
\newcommand{\project}{Explication du projet et des choix techniques}

% Main doc author (that would be you ;))
\newcommand{\authorname}{Corentin Didriche}
\newcommand{\authoremail}{cdidriche@numericube.com}

% Document information and version. Docrefs are N3-CUS-PRJ-TYP
% Where N3 is N3 ;)
% CUS is a customer trigram
% PRJ is a project trigram
% TYP is a type of doc (SRS, TEC, TST, ...)
\newcommand{\docref}{N3-GCA-EUR-TEC}

% Customer full name and logo
\newcommand{\customer}{NumeriCube}
\newcommand{\logoCustomerBW}{assets/BP-Logo-mono-black.png}

% Doc diffusion. If you don't want to include recipients, just comment them.
%\newcommand{\destA}{Firstname / Lastname of customer}
%\newcommand{\destB}{Firstname / Lastname of customer}

% Logos (EPS format please). Remove if you don't have.

% XXX TODO
% Texto
% \newcommand{\miGrado}{Grado en Ingeniería en Sonido e Imagen en Telecomunicación}
% Datos del tutor/es
%\newcommand{\departamentoTutor}{Departamento del tutor}
%\newcommand{\departamentoTutorB}{Departamento del cotutor}
% Datos de la facultada y universidad
%\newcommand{\miFacultad}{Escuela Politécnica Superior}
%\newcommand{\miFacultadCorto}{EPS UA}
%\newcommand{\miUniversidad}{\protect{Universidad de Alicante}}
%\newcommand{\miUbicacion}{Alicante}

% Metadata information to be used inside the PDF file.
\hypersetup{
pdfauthor = {\authorname~(\authoremail)},
pdftitle = {\doctitle},
}

%%
% Archivo de acrónimos
%%
\makeglossaries % Genera la base de datos de acrónimos
% Glossary and acronyms management
% See https://en.wikibooks.org/wiki/LaTeX/Glossary#Defining_symbols
% for more information.
%
% La forma de definir un acrónimo es la siguiente:
% \newacronyn{id}{siglas}{descripción}
% Donde:
% 	'id' es como vas a llamarlo desde el documento.
%	'siglas' son las siglas del acrónimo.
%	'descripción' es el texto que representan las siglas.
%
% Para usarlo en el documento tienes 4 formas:
% pepper your writing with \gls{mylabel} macros (and similar) to simultaneously insert your predefined text and build the associated glossary.
% \glsentryshort{id} - Añade solo las siglas de la id
% \glsentrylong{id} - Añade solo la descripción de la id
% \glsentryfull{id} - Añade tanto  la descripción como las siglas

\newacronym{n3}{N3}{NumeriCube}
\newacronym{cli}{CLI}{Command Line Interface}

\newglossaryentry{github}
{
    name={GitHub},
    description={
        The source management service (owned by Microsoft).
        See \url{https://www.github.com}
    }
}

\newglossaryentry{draft}
{
    name={Draft},
    description={
        A draft document is a document that is neither production-grade nor
        delivery-grade. A document whose reference is postfixed by the "\-DRAFT"
        mention is a document that has been generated during development process
        with misalignment between the repository revision and the moment it's been
        generated.
    }
}

% \newglossaryentry{real number}
% {
%   name={real number},
%   description={GROUMPF. include both rational numbers, such as $42$ and
%                $\frac{-23}{129}$, and irrational numbers,
%                such as $\pi$ and the square root of two; or,
%                a real number can be given by an infinite decimal
%                representation, such as $2.4871773339\ldots$ where
%                the digits continue in some way; or, the real
%                numbers may be thought of as points on an infinitely
%                long number line},
%   symbol={\ensuremath{\mathbb{R}}}
% }

\newglossaryentry{docker}
{
  name={Docker},
  description={include both rational numbers, such as $42$ and
               $\frac{-23}{129}$, and irrational numbers,
               such as $\pi$ and the square root of two; or,
               a real number can be given by an infinite decimal
               representation, such as $2.4871773339\ldots$ where
               the digits continue in some way; or, the real
               numbers may be thought of as points on an infinitely
               long number line},
  symbol={\ensuremath{\mathbb{R}}}
}
 % Archivo que contiene los acrónimos

%%%%%%%%%%%%%%%%%%%%%%%%%%%%%%%%%%%%%%%%%%%%%%%%%%%%%%%%%%%%%%%%%%%
%%%                    THE DOCUMENT ITSELF                      %%%
%%%%%%%%%%%%%%%%%%%%%%%%%%%%%%%%%%%%%%%%%%%%%%%%%%%%%%%%%%%%%%%%%%%
\begin{document}

% Números romanos hasta el mainmatter.
\frontmatter

% COVER PAGE
\input{include/cover_color} % Portada Color
\input{include/cover_bw} % Portada B/W

% A partir de aquí aplica los márgenes establecidos en configuracioninicial.tex
\restoregeometry

%%%%% PREAMBULO
% Incluye el .tex que contiene el preámbulo, agradecimientos y dedicatorias.
%%%%%%%%%%%%%%%%%%%%%%%%%%%%%%%%%%%%%%%%%%%%%%%%%%%%%%%%%%%%%%%%%%%%%%%%
% Plantilla TFG/TFM
% Escuela Politécnica Superior de la Universidad de Alicante
% Realizado por: Jose Manuel Requena Plens
% Contacto: info@jmrplens.com / Telegram:@jmrplens
%%%%%%%%%%%%%%%%%%%%%%%%%%%%%%%%%%%%%%%%%%%%%%%%%%%%%%%%%%%%%%%%%%%%%%%%

\chapter*{Executive Summary}
% \thispagestyle{empty}
This document describes the architecture and stratregy used to build the N3 demos website.
It's aimed at developpers and NumeriCube fellows in order to understand clearly our methodology of work.

Basically, we used:
\begin{description}
\item[Django] the Python framework with batteries included. Mostly used for API.
\item[Vue.js] as the main frontend framework (JS-based)
\item[Docker] for container management
\item[Docker Swarm] for container orchestration
\end{description}

\cleardoublepage %salta a nueva página impar



% Incluye después del archivo anterior el indice y lista de figuras, tablas y códigos.
\tableofcontents	% Índice
% \listoffigures		% Índice de figuras
% \listoftables		% Índice de tablas
% \lstlistoflistings	% Índice de códigos

% Inicia la numeración habitual.
\mainmatter

%%%%
% MAIN CONTENT. CAREFUL: SHELL EXTENSIONS ARE MANDATORY
%%%%
% Use the following line to automatically include all chapters.
% Harder to debug, though.
% \inputAllFiles{.}% from the current dir

% Use the following lines to include chapters one by one.
%!TeX root = ../main.tex
\chapter{Introduction}

Ce chapitre explique les généralités du document.

\section{Objectifs}

L’objectif de ce projet est de proposer un outil de détection visuelle pour l’indentification de numéros de série sur des dispositifs médicaux transparents, emballés.

\section{Domaine d’application}

L’outil devra être intégré des la chaine de production d’Eurosolicone.
Cet outil devra fonctionner de manière autonome, et être interfacé avec l’outil de gestion de production propre à Eurosilicone. Il devra respecter toute une série de contraintes techniques, et faire l’objet d’une validation.
La détection des numéros de serie doit être constante et répétable. De plus, le processus de traitement doit répondre à des contraintes temporelles (- de 10 secondes).

\section{Contraintes techniques}

%!TeX root = ../main.tex
\chapter{Matériel}

Ce chapitre explique les généralités du document.

\section{Boîte}

La boite doit être conçue de façon à respecter les conditions suivantes :
\begin{itemize}
    \item Être intégrable sur le plan de travail dédié dans la chaine de production Eurosilicone : Les dimensions totales de la boite doivent tenir sur une paillasse.
    \item Être relativement compacte : L’objet doit être le plus petit possible, sans en altérer son fonctionnement.
    \item Permettre de réaliser la lecture des numéros de série de tous les types de dispositifs médicaux, indépendamment de la taille : Une fenêtre de verre doit être prévue pour permettre au dispositif de prise de vue de faire l’acquisition des données visuelles, ainsi qu’un dispositif d’éclairage permettant la prise de vue dans des conditions exploitables.
    \item Être facilement utilisable par les opérateurs/opératrices : Rejoint l’aspect ergonomie, doit permettre la manipulation des prothèses facilement sur le lecteur.
    \item Ne pas être sensible aux solvants/nettoyants utilisés dans la chaine de production : La boite devra être composée que de matériaux qui ne réagissent pas les détergents utilisés en chaine de production.
\end{itemize}

\section{Conception de la boite}

PHOTO DE LA BOITE

Les matériaux utilisés seront :
\begin{itemize}
    \item Du plexiglass noir pour les parties « externe » de la boîte : Ce sera le composé principal de la structure, résistant aux détergents.
    \item Une vitre de verre pour la partie haute, permettant au système d’acquisition de lire le numéro de la prothèse.
    \item Barres de LEDs 10 watts, 1000 Lumens. Orientables pour eviter l'éblouissement, ainsi que pour permettre l'éclairement optimal pour la visualisation des numéros de série.
\end{itemize}

D’un point de vue technique sur la conception de la boite, il faut prendre en compte :
\begin{itemize}
    \item La plaque de verre doit être d’une taille suffisante pour permettre de visualiser la boite de la plus grande prothèse ainsi que les prothèses en sachet. Soit 20cmx20cm*4mm.
    \item L’éclairage doit être assez puissant et régulier pour visualiser correctement les numéros de série avec le dispositif de prise de vue. Après test, on a déterminé qu’il était préférable d'opter pour un éclairage rasant. En effet, une lumière rasante permet d’offrir des résultats plus réguliers, et surtout de fournir des prises de vues avec de fort contrastes au niveau des numéros de série.
    \item La camera à deux contraintes fortes : La distance de mise au point, et l’angle de champ de la caméra.
    La distance de mise au point est propre au capteur/objectif, il doit être d’assez bonne facture pour permettre une mise au point à courte distance.
    L’angle de champ est fonction de la focale, il doit être assez élevé pour visualiser toute la zone a « scanner » environ 10cmx10cm, de ce fait, la distance entre la camera et la plaque de verre doit être fonction de l’angle de champ de la caméra. Évidemment, l’angle de champs doit être élevé, pour minimiser la distance caméra/plaque de verre, afin de rendre la boite plus compacte.
    \item L’écran de retour : Sur cet écran, on a un retour de la lecture des numéros de série, qui permettra à l’opérateur de visualiser le bon déroulement de la lecture des numéros de série.
    \item Le mini PC : Le mini PC se chargera de piloter le dispositif de prise de vue, d’interfacer la saisie clavier avec l'Arduino DUE, ainsi que d’exécuter toute la partie algorithmique pour : la détection du produit, le traitement de l’image, la détection et l’identification des caractères gravés.
    \item L'Arduino DUE : L'Arduino se chargera de générer les trames du clavier, selon les caractères détéctés par le système.
\end{itemize}


\section{Mini PC et Caméra}

\begin{itemize}
    \item Le mini PC est un NUC d’Intel (dimensions L x W x H : 11,5 x 11,1 x 5,1 cm) NUC8I3BEH, disposant d’aucune pièce mécanique (SSD). Celui-ci va être assez véloce pour faire tourner l’algorithme, et gérer toutes les E/S hardware.
    \item La caméra est une Basler Ace acA5472-17um USB3 Mono, 20Mpx, c’est une caméra USB fonctionnant sous linux (SDK).
    \item Objectif Fujinon CF16ZA-1S F1.8/16mm.
\end{itemize}

\section{Mini écran de retour}

Le mini écran de retour sera simplement connecté en HDMI au mini PC, et intégré dans la boite sur une paroi. C'est un Waveshare 5 Pouces LCD HDMI 800 * 480 Haute résolution.

%!TeX root = ../main.tex
\chapter{Conception algorithmique}

Dans ce chapitre,

\section{Déclenchement de la prise de vue et identification de la pastille}

Pour déclencher la prise de vue, il faut au préalablement créer la partie détection d’une prothèse en place.
Pour ce faire :
\begin{itemize}
    \item Analyse des frames de la camera, toutes les X ms.
    \item Détection et localisation du cercle des pastilles contenant les numéros de série.
\end{itemize}
Libraires utilisées : pyPylon, SDK caméra, Keras

\section{Traitement de l’image}

Après détection de la pastille via deep learning, il faut nécassairement traiter l'image qui va nous permettre de rendre les numéros de série sur les images exploitable. L’image va suivre un pipeline de traitement, générique pour toutes les images (et donc tous les produits scannés), qui va permettre de rendre l’information visuelle exploitable.
Le pipeline se décompose de la façon suivante :
\begin{itemize}
    \item Découpage et redimensionnement de la photo sur la pastille pour s’affranchir de données visuelles sans intérêt, et accélérer les temps de traitement.
    \item Vérification du format de l'image, conversion de l’image en Noir & Blanc si besoin.
    \item Passe d’égalisation adaptative de l’histogramme
\end{itemize}

Libraires utilisées : OpenCV

\section{Redressement de l'image et détection des zones de texte}

Après le traitement d'image, nous devons redresser l'image, et détecter les zones de texte. Nous allons justement utiliser la détéction de texte pour redresser l'image. L'algorithme va boucler sur la photo, de façon a détecter les zones de texte. Si la détéction n'est pas concluante, alors on tourne l'image de X degrés. On efféctue cette opération autant de fois que nécéssaire (généralement sur 360 degrés).
Le pipline :
\begin{itemize}
    \item Détection zones de texte.
    \item Vérification du nombre de détection.
    \item Rotation de l'image de X degrés.
    \item Découpage et redimensionnement de la photo sur les zones de texte de la pastille pour s’affranchir de données visuelles sans intérêt, et accélérer les temps de traitement.
\end{itemize}

Libraires utilisées : OpenCV, keras

\section{Détection des caractères des numéros de série}

Troisième et derniere étape d'inférence : La détéction des caractères.
Pour chaque zone de texte détectée :
\begin{itemize}
    \item Détection et identification des caractères.
    \item Re-constitution des numéro de série.
    \item Envoi des numéros de série via liaison série vers l'Arduino.
\end{itemize}

Libraires utilisées : keras, serial

\section{Transmission et saisie clavier sur la machine client}

Pour des raisons techniques, il est impossible d’assurer une transmission filaire PC to PC via USB. Les port USB d’un ordinateur ne peut pas être programmés pour générer et simuler des périphériques USB. On va donc obligatoirement passer par une solution hardware programmable, qui va simplement faire le lien entre les deux PC via USB, tout en se comportant comme un clavier pour le PC client.
Cette solution hardware est l'Arduino DUE.

L’Arduino DUE se compose de 2 ports micro USB distincts (un natif, et un programmable):
\begin{itemize}
    \item Un Atmega16U2 (microcontrôleur), qui va assurer la communication avec le PC client, et la simulation des touches clavier.
    \item Un Atmel SAM3X (processeur ARM) qui va assurer la communication avec le PC source, en recevant les données correspondant au numéro de série.
    \item Rotation de l'image de X degrés.
    \item Découpage et redimensionnement de la photo sur les zones de texte de la pastille pour s’affranchir de données visuelles sans intérêt, et accélérer les temps de traitement.
\end{itemize}

L’alimentation de l’Arduino DUE se fait lorsque l’un des deux ports usb est connecté. On peut aussi l’alimenter en externe, via prise d’alimentation. Étant donné que l’Arduino est indépendant, il se comporte comme un périphérique « Plug and Play », même en cas de coupure de courant, celui-ci ne nécessite aucune manipulation particulière. L’alimentation USB simple suffira.

Partie mini PC : La librairie serial de python3 est utilisée pour communiquer et transmettre les caractères clavier à générer sur l’Arduino.

Partie Arduino : Les lignes de code qui vont transformer les caractères reçus par le mini PC, sont enregistrées dans l’EEPROM de l’Arduino. L’EEPROM rend l’Arduino insensible a la coupure de courant. Seul le code présent dans la mémoire de l’Arduino est exécuté.
Les caractères lus sur l’image du mini PC sont transmis à l’Arduino via liaison USB/série. L’Arduino va générer et envoyer presque instantanément les trames correspondantes à ces caractères, en ce comportent de façon totalement transparente comme périphérique clavier pour le PC client.

\section{Librairies requises : Résumé}

Résumé des libraires/import requis(es) pour le projet :
Python3, Opencv, pypylon (sdk basler/swig/gcc), numpy, time uuid, tkinter, requests, http, urllib, math, imutils, threading, serial.

Pour la partie camera/SDK Basler, se référer au github Basler (https://github.com/basler/pypylon)
Toutes les autres librairies sont à installer via pip ou incluses nativement dans Python3.
Pour l’Arduino, le code est également sur le github Numericube, sur le repo dédié au projet, mais n’a logiquement pas besoin d’être re-flashé.
En ce qui concerne l’OS, c’est un Ubuntu qui est installé sur le NUC (Mini PC).
Enfin le fichier NodeMapSAVE.pfs est le fichier dans lequel est enregistré la configuration caméra, nécessaire après redémarrage du système, pour une parfaite autonomie. Il va cependant dépendre des paramètres lumineux environnementaux, une version « finale » sera donc enregistrée chez le client. Puisque les réglages caméra dépendent de l’éclairage et environnement...

%!TeX root = ../main.tex
\chapter{Découpage des fonctions}

Dans ce chapitre, vous retrouverez en détails le découpage des fonctions du projet.

\section{Camera.py}
\subsection{loadConf (self) :}
Fonction permettant de charger le fichier de configuration (les paramètres comme exposure time, etc...) de la caméra. Ces paramètre sont définis dans un fichier au format .pfs (NodeMap.pfs).
Input : fichier de configuration .pfs
Output : Message configuration bien chargée.

\subsection{saveConf (self) :}
Fonction permettant d’enregistrer les paramètres de configuration de la caméra, dans un fichier .pfs. Les paramètres caméra sont réglé préalablement dans l’outil Pylon, qui permet un visuel temps réel de la caméra et de ses réglages.
Input : Nom fichier de configuration .pfs (self.conf)
Output : Message configuration ok + fichier de configuration

\subsection{grabbingImage (self) :}
Fonction permettant la capture d’une image a l’instant T.
Input : Initialisation de la camera (self.instant_camera)
Output : currImg (Image opencv plein format), resizeImg (Image opencv Full HD)

\subsection{saveImage (self, currImg, resizeImg) :}
Fonction permettant d’enregistrer les images capturées par la caméran sur le disque. Enregistre les images plein format et full HD, à la racine courante (ou ailleurs...), avec un nom généré par UUID.
Input : Image opencv plein format, image opencv full HD
Output : Message « Images saved »

\subsection{generateUUID :}
Fonction permettant de générer un UUID, pour le nommage de fichiers (principalement images).
Output : UUID sous forme de string, généré a l’instant de l’appel T. (Vu que c’est généré selon le timestamp...)

\subsection{showImage (self) :}
Fonction permettant d’afficher l’image. Utilisée et exploitée seulement pour le dev...
Input : Image opencv
Output : Fenêtre python avec l’image affichée.

\section{Detection_instance.py}
\subsection{get_chip_area (self) :}
Fonction permettant de cropper et égaliser l’image autour de la pastille détectée.
Input : Image (capturée par la caméra), self.chip_crop (coordonnées de la bounding box détecté)
Output : Booléen (True si détection, false si pas de détection...), image (image croppée autour de la pastille)

\subsection{get_text_area (img, out_boxes) :}
Fonction permettant de cropper l’image autour des textes détectés. Il y en a 3 par image.
Input : Image (provenant de l’image croppée get_chip_area), out_boxes (coordonnées des bounding boxes détectés autour des zones de texte)
Output : Image1, Image2, Image3 (Chaque image représentant une zone de texte)

\subsection{get_text_orientation (self) :}
Fonction permettant de détecter l’orientation des textes, et donc, de définir l’orientation de l’image.
Cette fonction va appeler la fonction get_texte_area, et boucler sur celle-ci de façon à détecter et isoler les 3 champs de textes des photos de pastilles, selon une rotation de l’image definit.
Input : Image (croppée de la pastille), int (entier d’incrément de la rotation de l’image)
Output :

\subsection{read_text (self) :}
Fonction permettant de lire les caractères sur chacune des zones de texte.
Input : Image de texte 1, image de texte 2, image de texte 3
Output : 3 listes de string, représentant chacune, les caractères d’une des 3 zones de texte.

\section{Image.py}
\subsection{openImage (self, filename) :}
Fonction permettant l’ouverture d’une image.
Input : Chemin d’accès de l’image
Output : Image opencv

\subsection{addSerialNumber (self, serialNumber) :}
Fonction permettant d’incruster le numéro de série lu, sur l’image retour visible par l’opérateur.
Input : Image (image sans le numéro de série incrusté)
Output : Image (image avec numéro de série incrusté)

\section{Gui.py}
\subsection{run (self) :}
Fonction d’exécution de thread. Boucle en permanence sur la fonction displayImage().

\subsection{update_txt (self) :}
Fonction d’affichage texte message « Upate.txt », utilisée lors du dev.
loadImage (self, filename, resize=None) :
Fonction permettant de charger l’image qui sera affichée.
Input : filename de l’image
Output : Image tkinter

\subsection{displayImage (self) :}
Fonction permettant l’affichage de l’image, dans une fenêtre tkinter (gui...).
Input : Image tkinter (Image output loadImage)
Output : Affichage de la GUI avec l’image tkinter

\subsection{onClose (self) :}
Fonction de stop qui « tue » la fenêtre lors de la fermeture de celle-ci.

\section{Keyboard.py}
\subsection{openAndListen (self) :}
Fonction permettant d’initialiser et de tester la connexion série entre le PC qui permet de détecter les numéros de série et l’Arduino DUE.
Input : serial_device
Output : Ouverture liaison série

\subsection{isAvailable (self) :}
Test la disponibilité du port série de l’Arduino (et donc, test la conexion PC/Arduino)
Input : serial_device
Output : booléen (isAvailable)

\subsection{connect (self) :}
Fonction qui permet d’initialiser la connexion entre le PC et l’arduino DUE.
Input : self.serial_device
Output : self.serial_device

\subsection{stopOnJoin (self) :}
Fonction qui sert à couper la connexion série entre le PC et l’arduino DUE.
Input : self.serial_device
Output : self.serial_device

\subsection{send (self, data) :}
Fonction qui permet d’envoyer les données PC via la liaison série vers l’arduino DUE. Ces données sont par la suite converties en signal « clavier » par l’Arduino vers le PC client.
Input : self.serial_device, data (liste de caractères)
Output : données encodées sur port série

\section{Yolo.py}
Un fichier pour chaque détecteur. Il faudra surement factoriser par la suite...
\subsection{_get_class (self) :}
Fonction qui permet de récupérer le nom de des classes à partir du fichier dédié pour.
Input : class_file path
Output : liste de string nom des classes

\subsection{_get_anchors (self) :}
Fonction permettant
Input : anchor_file path
Output : numpy array des anchors

\subsection{generate (self) :}
Fonction permettant de charger les model et les fichiers de poids.
Input : Fichiers de poids, fichier de classe, fichier anchors...
Output : Tensor initialiser pour inférer !

\subsection{detect_img (self, image) :}
Fonction permettant d’exécuter l’inférence sur une image donnée.
Input : Tensor chargé, image a inférer.
Output : is_detected (bool, True si detection il y a), out_boxes (coordonnées des bounding box), out_scores (score de confiance), out_classes (classe correspondante)

\subsection{Utils Yolo3 :}
Le dossier Yolo3 contient 3 fichiers : init.py, model.py, utils.py. Ces fichiers contiennent une série de fonctions nécessaire à l’exécution de yolo3.
\begin{itemize}
    \item Model.py : Contient toute la définition des architectures yolo et tiny yolo.
    \item Utils.py : Contient des fonctions principalement de pré-processing, comme le redimensionnement des images, la modification des images (pour data augmentation par exemple), etc...
\end{itemize}

\subsection{Dépendances et fichier de configuration nécessaires pour Yolo3 :}
Chaque inférence doit avoir ses propres fichiers de configuration :
XXXX : Nom de chaque étape (détection pastille / détection texte / détection caractères)
\begin{itemize}
    \item Classes_XXXX.txt : Fichier texte listant les classes.
    \item Tiny_yolo_anchors_XXXX.txt : Fichier texte listant les dimensions des anchors (6 anchors, width,height)
    \item Tiny_yolo_weights_XXX.h5 : fichier de poids
\end{itemize}

\section{Main.py}
Fichier d’exécution principal, contenant l’appel des fonctions...
Le fichier main.py est découpé en plusieurs classes.
Class Main (Thread) : Thread gérant la partie « processing ».
Class ImplantBox : Thread gérant la partie GUI.

\section{ArduinoProgram.ino}
Fichier de code du microcontrôleur Arduino. Il se décompose de la façon suivante :
\subsection{Setup () :}
Fonction permettant la configuration de l’Arduino. Initialisation des pins/port, configuration liaison série, et configuration du « mode » génération touches clavier (azerty fr).
\subsection{Loop () :}
Fonction permettant l’envoi et la génération des données clavier.
Principe :
\begin{itemize}
    \item Le PC envois une string finissant par un retour à la ligne
    \item L’Arduino reçoit une string, comprend que c’est la fin de la string avec le retour à la ligne
    \item L’Arduino renvoi la string reçu pour vérifier que les données envoyées par le PC correspondent à celles reçu par l’Arduino
    \item Génération et envoi des données « clavier » sur le port dédié à l’envoie des données PC client
\end{itemize}
Au niveau des ports :
\begin{itemize}
    \item Le port natif, relié à l’Atmel (CPU) : PC client
    \item Le port programming (coté alimentation) : PC source (donc relié au NUC)
\end{itemize}


\section{Gestion de la GUI (activation / désactivation)}
Pour des raisons pratiques de ressources et de minimisation des potentiels risques de problèmes, nous désactiverons la GUI. Il suffira de se logger avec User/Password au démarrage de la machine.

Pour information, les commandes d’activation/désactivation GUI sont :
Pour désactiver la GUI au boot :
\begin{lstlisting}[style=Latex-color]
sudo systemctl set-default multi-user.target
\end{lstlisting}

Pour réactiver la GUI au boot :
\begin{lstlisting}[style=Latex-color]
sudo systemctl set-default graphical.target
\end{lstlisting}

\section{Configuration de la camera via app PylonViewer}
Pour la gestion de la configuration de la caméra, on utilise l’application PylonViewerApp.
Le fichier exécutable se trouve dans /opt/pylon5/bin/PylonViewerApp
L’application permet de visualiser l’image caméra, ainsi que des régler ses paramètres.
Une fois l’application fermée, les paramètres camera restent sur les derniers réglages.
De ce fait, en exécutant la fonction d’enregistrement de configuration, on obtient le fichier NodeMapSAVE.pfs

%!TeX root = ../main.tex
\chapter{Entraînement}

Dans ce chapitre, vous retrouverez en détails les procédures pour l'entrainements des fichiers de poids Tiny Yolo.

\section{Yolo et Tiny Yolo sous Keras}

Le code des entraînements se trouve dans le dossier gcaesthetics-implantbox/train/keras-yolo3-master.
Les entrainements se font classiquement avec l'envoi et l'execution d'un container Docker sur Paperspace.
Les fichiers importants a modifier pour lancer/customiser les entraînements sont :

\subsection{run.sh}
Comme dit précement, le lancement de l'entraînement se fait sur container Docker, envoyé sur des machines Paperspace.
Le fichier run.sh va executer les commandes ligne par ligne.

\subsection{train.py}
C'est le script de lancement de l'entraînement.
Concrètement, les seuls paramètres a modifier dans ce fichiers sont :
\begin{itemize}
    \item anchors_path : Chemin d'accés du fichiers d'anchors.
    \item classes_path : Chemin d'accés du fichier de classes.
    \item input_shape : Taille des "input", comprendre la taille des photos en entrée. Cette valeurs doit forcément être un multiple de 32.
    \item Les paramètres d'entrainements plus bas dans le fichier au besoin (comme le batch_size, l'archi, etc...)
\end{itemize}

\subsection{convert_to_voc.py}
Permet de convertir les annotations from supervisely to keras yolo dans un fichier de sortie /artifact/train.txt, qui sera utilisé dans train.py.
Il est généré avant chaque lancement d'entraînement.
Il prends en entrée :
\begin{itemize}
    \item image_path : Chemin d'accés du dossier d'image d'un dataset donné.
    \item annot_path : Chemin d'accés du dossier d'annotations d'un dataset donné.
\end{itemize}

\subsection{Datasets}
Tous les datasets sont disponibles dans le bucket S3.
Ils sont synchronisés dans la partition /storage/.
Evidemment, tous ces datasets sur le projet Eurosilicone sont provisoires,

\subsubsection{Détection des pastilles}
\begin{itemize}
    \item pastilles : dataset constitué avec la 1ère caméra couleur, pour la détection de pastilles, sans égalisation ni cropping/redressement.
\end{itemize}

\subsubsection{Détection de textes}
\begin{itemize}
    \item textes : dataset constitué avec la 1ère caméra couleur/focale, redressées
    \item textes_equal : dataset « textes », avec les images égualisées
    \item textes_20mpx : dataset constitué avec la caméra monochrome 20mpx, images redressés
    \item textes_20mpx_equal : dataset « textes_20mpx », contenant les même images égualisées
\end{itemize}

\subsubsection{Détection de caractères}
\begin{itemize}
    \item lettres_equal : dataset constitué avec la caméra monochrome 20mpx. Comprend les images croppées et redréssées autour des pastilles.
    \item lettres_equal_crop : dataset « lettres_equal », avec cropping autour des textes
\end{itemize}

\subsection{Resultats de l'entraînement}

En sortie d'entraînement, nous avons un fichier de poids .h5
Ce fichier est généré dans le dossier /artifacts sur Paperspace.

\subsection{Evaluation de l'entraînement}

Pour la phase d'evaluation, nous allons executer l'inférence sur un dossier de photos jamais vues par l'algorithme.
Pour des questions pratiques de developpement, j'execute les inferences en local (commande : make debug), mais il est possible de l'executer sur Paperspace a la suite d'un entrainement...

\subsection{yolo_video.py}

Fichier permettant l'execution des inference. Les commandes a lancer sont :
\begin{lstlisting}[style=console]
    Air-de-Corentin:keras-yolo3-master cdidriche$ : make debug
    root@75120bff2ed0:/paperspace# : python3 yolo_video.py --image
    root@75120bff2ed0:/paperspace# : ./images_to_infer
    [.......]
\end{lstlisting}

\begin{itemize}
    \item make debug : Execution du container en local
    \item python3 yolo_video.py --image : Permet de lancer le script d'inférence
    \item ./images_to_infer : est le path contenant les images a inférer. Le script va boucler sur toutes les images du dossier en question. Un fichier inference_result.txt est généré dans ./model_data. Il va nous servir pour le calcul du mAP derrière, et donc, d'avoir une métrique d'évaluation significative
\end{itemize}

Par défault, les chemins sont en "dur", un dossier model_data qui contient les fichiers .h5, classes.txt, anchors.txt dans le fichier yolo.py.
C'est pas très smart ni user-friendly, mais au moins ca permet de ne pas faire d'erreurs de dossier/fichiers quand on utilise un même detecteur pour 3 cas différents...

\subsection{Calcul du mAP}

Pour calculer le mAP, on utilise une série de script qui permet d'avoir une evaluation "générique".
Le code du calcul de mAP se situe dans ./train/mAP

Avant de lancer le calcul du mAP, il faut génerer pour chaque image, un fichier "ground-truth" et un fichier "predicted objects".
Pour les générer simplement, un scripts sont a disposition pour convertir les annotations : convert_keras-yolo3.py
\begin{itemize}
    \item Depuis le fichier "train.txt" (sortie du script convert_to_voc.py), vers le dossier "ground-truth".
    \item Depuis le fichier "inference_result.txt" (sortie du script d'inference Yolo), vers le dossier "predicted".
\end{itemize}

Le script pour fonctionner a besoin des fichiers :
\begin{itemize}
    \item class_list.txt : Fichier contenant la liste des classes, disponible dans ./extra
    \item train.txt OU inference_result.txt : Fichier contenant les annotations, disponible dans
\end{itemize}

\begin{lstlisting}[style=console]
    Air-de-Corentin:mAP-master cdidriche$ cd extra/
    Air-de-Corentin:extra cdidriche$ python3 convert_keras-yolo3.py -o ../ground-truth/ -r --gt ../ground-truth/train.txt
    [.......]
    Air-de-Corentin:extra cdidriche$ python3 convert_keras-yolo3.py -o ../predicted/ -r --pred ../predicted/inference_result.txt
    [.......]
    Air-de-Corentin:mAP-master cdidriche$ python3 main.py
    100.00% = m AP
    100.00% = text AP
    mAP = 100.00%
\end{lstlisting}

En sortie, on dispose de tout un ensemble de graphique et meusure representant le mAP globale, par classe, etc...
Ces résultats sont diponibles et consultables dans le dossier /mAP-master/results


%%%%
% CONTENIDO. Appendices
%%%%
\appendix % Inicio de los apéndices
%!TeX root = ../main.tex
\chapter{Commit history}
% \IfFileExists{../docgen/commit_history.tex}{
\input{./docgen/commit_history.tex}
% }{
% Commit history not available.
% }

% \newcommand\commit[2]{\node[commit] (#1) {}; \node[clabel] at (#1) {\texttt{#1}: #2};}
% \newcommand\ghost[1]{\coordinate (#1);}
% \newcommand\connect[2]{\path (#1) to[out=90,in=-90] (#2);}
%
% \begin{tikzpicture}
% \tikzstyle{commit}=[draw,circle,fill=white,inner sep=0pt,minimum size=5pt]
% \tikzstyle{clabel}=[right,outer sep=1em]
% \tikzstyle{every path}=[draw]
% \matrix [column sep={1em,between origins},row sep=\lineskip]
% {
% \commit{d764b48}{added plaintext version in markdown} & \\
% \commit{54ba4b2}{release 2014-01-25} & \\
% \commit{c589395}{Merge branch `master'} & \\
%  & \commit{9f9c652}{Remove holdover from kjh gh-pages branch} \\
% \commit{b3bd158}{exclude font files} & \ghost{branch1} \\
% \commit{63268c1}{micro-typography} & \\
% };
% \connect{63268c1}{b3bd158};
% \connect{63268c1}{branch1};
% \connect{branch1}{9f9c652};
% \connect{b3bd158}{c589395};
% \connect{9f9c652}{c589395};
% \connect{c589395}{54ba4b2};
% \connect{54ba4b2}{d764b48};
% \end{tikzpicture}


%%%%
% BIBLIOGRAPHY AND GLOSSARY
%%%%
% \nocite{*} %incluye TODOS los documentos de la base de datos bibliográfica sean o no citados en el texto
% \bibliography{bibliografia/bibliografia} % Archivo que contiene la bibliografía
% \bibliographystyle{apacite}

\printglossary[style=altlist,type=\acronymtype,title={Terms, definitions, and abbreviations}]
\glsaddallunused

\input{include/cover_back}

\end{document}
